% !TEX TS-program = xelatex
%
% Created by Gerson Sunyé on 2020-05-14.
% Copyright (c) 2020 .
\documentclass{article}

\usepackage{polyglossia}
\usepackage{hyperref}
\usepackage{science}

\hypersetup
{
  pdftitle   = {Title},
  pdfsubject = {Subject},
  pdfauthor  = {Gerson Sunyé}
}

\title{Package science.sty examples}
\author{Gerson Sunyé}

\begin{document}

\maketitle

\begin{abstract}
    Abstract
\end{abstract}

\section{Abbreviations}

The work of Sigmund~\etal, \eg, ``Man and Father'' teach us the The Legend of a Conqueror, \ie, the ``Oedipus legend''.
In speaking of Oedipus ``legend'', I am following William Bascom,
who distinguishes between myth, legend, and folktale \wrt characters. (divine vs. human), time, setting,  \etc


\section{Using listings.sty environments}

\subsection{OCL}

\begin{ocl}
context Person
pre: 
	age > 0 and 
	not firstName.oclIsUndefined() and
	children->forAll(each | each.isValid())
\end{ocl}



\subsection{ATL}

\begin{atl}
rule relationnalContextToObjectContext {
	from
		r : RDBMSMM!RDBMSModel
	to
		c : ClassMM!ClassModel (
			classifier <- r.table->union(r.getAllDataTypes()
				->collect(dt | thisModule.typeToPrimitiveDataType(dt)))
		)
	do {
		r.debug('relationnalContextToObjectContext');
	}
}
\end{atl}
\end{document}